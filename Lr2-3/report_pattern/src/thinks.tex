\section{Выводы}
PATRICIA-дерево является высокоэффективной структурой данных для реализации словарей со строковыми ключами. Его ключевое преимущество — временная сложность операций $O(K)$ (где $K$ — длина ключа), которая не зависит от общего количества элементов $N$ в структуре. Это обеспечивает стабильно высокую производительность даже на очень больших наборах данных, в отличие от сбалансированных деревьев (например, \texttt{std::map}), чья сложность $O(K \cdot \log N)$ деградирует с ростом $N$.

В ходе выполнения лабораторной работы была реализована программа-словарь на основе PATRICIA-дерева. Основной задачей было создание производительной структуры, превосходящей стандартные аналоги на больших объемах данных. Это было достигнуто благодаря реализации ключевого принципа PATRICIA: навигации по дереву на основе отдельных битов ключа, что позволяет избежать дорогостоящих полных сравнений строк на каждом шаге спуска. Этот опыт наглядно демонстрирует, как выбор специализированной структуры данных, учитывающей природу хранимой информации, может кардинально повысить производительность, и знакомит с устройством одного из классических и эффективных алгоритмов для работы со строковыми ключами.

Также был составлен отчет при помощи системы \TeX, которая позволяет автоматизировать процесс создания качественной технической документации.
\pagebreak