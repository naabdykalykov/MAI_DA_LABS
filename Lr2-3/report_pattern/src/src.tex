\section{Описание}

Для решения задачи была реализована структура данных \textbf{PATRICIA-дерево} — пространственно-эффективная реализация префиксного дерева (Trie). В отличие от обычного Trie, которое создает узел для каждого символа, PATRICIA \textbf{пропускает общие префиксы и создает узлы ветвления только в точках первого битового различия между словами.} Этот принцип построения "сжатого" дерева обеспечивает значительную экономию памяти. Реализация поддерживает регистронезависимые операции путем приведения ключей к нижнему регистру
Дерево состоит из \textbf{листовых узлов}, хранящих пару ключ-значение, и \textbf{внутренних узлов}, которые содержат только индекс бита для навигации.

Основные операции основаны на побитовом анализе ключей:
\begin{itemize}
    \item \textbf{Поиск (Find):} Спуск по дереву на основе битов ключа до листового узла с последующей полной проверкой ключа.
    \item \textbf{Вставка (Insert):} Поиск ближайшего ключа, определение первого различающегося бита и вставка нового внутреннего узла ветвления на этом уровне.
    \item \textbf{Удаление (Remove):} Поиск узла, его удаление вместе с родительским узлом ветвления и "поднятие" соседней ветви для сохранения структуры.
\end{itemize}

Для команд \texttt{! Save} и \texttt{! Load} реализован механизм сериализации. Он использует рекурсивный обход для записи дерева в бинарный файл с маркерами типов узлов и "магическим числом" для проверки целостности файла.
\pagebreak

\section{Исходный код}
Программа написана на языке C++ с использованием стандартной библиотеки. Код структурирован модульно и разделен на три логических блока: вспомогательные утилиты, класс \texttt{Patricia}, реализующий основную структуру данных, и набор функций-обработчиков команд.

Основная логика инкапсулирована в классе \texttt{Patricia}, а взаимодействие с пользователем происходит в функции \texttt{main} через вызовы функций-обработчиков.

\begin{longtable}{|p{6cm}|p{9cm}|}
\hline
\rowcolor{lightgray}
\multicolumn{2}{|c|} {Структура программы}\\
\hline
\texttt{class Patricia} & Основной класс, инкапсулирующий PATRICIA-дерево. Содержит приватную структуру \texttt{Node} для узлов дерева и реализует всю логику работы со словарем. \\
\hline
\texttt{int main()} & Главная функция. Организует цикл чтения команд из стандартного ввода, вызывает соответствующие функции-обработчики для каждой команды и отлавливает возможные исключения. \\
\hline
\texttt{handle\_add, handle\_remove, handle\_search, handle\_bang\_command} & Функции-обработчики команд. Разбирают строку с командой, извлекают аргументы (слово, число, путь к файлу) и вызывают соответствующие публичные методы класса \texttt{Patricia}. \\
\hline
\texttt{Patricia::insert, remove, find} & Публичные методы класса \texttt{Patricia}, реализующие основные операции словаря: добавление, удаление и поиск ключа в дереве. \\
\hline
\texttt{Patricia::save, load} & Публичные методы, отвечающие за сериализацию (сохранение) и десериализацию (загрузку) всего дерева в компактный бинарный формат. \\
\hline
\texttt{to\_lower, get\_bit} & Вспомогательные функции. \texttt{to\_lower} приводит строку к нижнему регистру для регистронезависимых операций, а \texttt{get\_bit} позволяет получить значение конкретного бита в строке, что является основой для навигации по дереву. \\
\hline
\end{longtable}

\pagebreak

Структура для хранения данных о тексте определена следующим образом:
\begin{lstlisting}[language=C++]
#include <string>
#include <cstdint>
#include <memory>
struct Node {
std::string key;
uint64_t value;
int bit_index;

std::shared_ptr<Node> left;
std::shared_ptr<Node> right;

Node(const std::string& k, uint64_t v);
Node(int b_idx);

bool is_leaf() const;
};
\end{lstlisting}
\pagebreak

\section{Консоль}
Демонстрация работы программы: компиляция исходного кода, просмотр содержимого тестового файла \texttt{input.txt}, содержащего последовательность команд для словаря, и запуск исполняемого файла с перенаправлением ввода для их обработки. Для наглядности сверхдлинные строки в выводе команды \texttt{cat} были сокращены.
\begin{alltt}
\fontfamily{pcr}\selectfont
laysou@DESKTOP-QPGEK53:/mnt/c/Users/abdyk/Desktop/Lr2$ g++ -std=c++17 -O2 -o patricia_dict patricia_dict.cpp
laysou@DESKTOP-QPGEK53:/mnt/c/Users/abdyk/Desktop/Lr2$ cat input.txt
+ a 1
+ A 2
+ aaaaa...aaaa 18446744073709551615
aaaaa...aaaa
A
- A
a
laysou@DESKTOP-QPGEK53:/mnt/c/Users/abdyk/Desktop/Lr2$ ./patricia_dict < input.txt
OK
Exist
OK
OK: 18446744073709551615
OK: 1
OK
NoSuchWord
\end{alltt}
\pagebreak
