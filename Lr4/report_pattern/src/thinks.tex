\section{Выводы}
Z-алгоритм является высокоэффективным методом поиска, демонстрируя линейную временную сложность, которая не зависит от структуры текста или количества частичных совпадений. Однако его область применения специализирована на точном поиске фиксированных образцов и не подходит для более сложных задач, таких как поиск с использованием регулярных выражений.

В ходе выполнения лабораторной работы был реализован Z-алгоритм для поиска одного образца в тексте, состоящем из слов. Ключевой задачей стало достижение линейной производительности, в отличие от наивного подхода с квадратичной сложностью. Это было достигнуто путем реализации основной оптимизации Z-алгоритма — использования границ самого правого Z-блока для избежания повторных сравнений. Этот опыт наглядно демонстрирует, как выбор правильной алгоритмической стратегии может кардинально снизить временные затраты, и знакомит с устройством одного из классических и эффективных алгоритмов обработки строк.

Также был составлен отчет при помощи системы \TeX, которая позволяет автоматизировать процесс создания качественной технической документации.
\pagebreak